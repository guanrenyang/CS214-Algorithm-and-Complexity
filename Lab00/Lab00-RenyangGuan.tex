\documentclass[12pt,a4paper]{article}
\usepackage{ctex}
\usepackage{amsmath,amscd,amsbsy,amssymb,latexsym,url,bm,amsthm}
\usepackage{epsfig,graphicx,subfigure}
\usepackage{enumitem,balance}
\usepackage{wrapfig}
\usepackage{mathrsfs,euscript}
\usepackage[usenames]{xcolor}
\usepackage{hyperref}
\usepackage[vlined,ruled,linesnumbered]{algorithm2e}
\hypersetup{colorlinks=true,linkcolor=black}

\newtheorem{theorem}{Theorem}
\newtheorem{lemma}[theorem]{Lemma}
\newtheorem{proposition}[theorem]{Proposition}
\newtheorem{corollary}[theorem]{Corollary}
\newtheorem{exercise}{Exercise}
\newtheorem*{solution}{Solution}
\newtheorem{definition}{Definition}
\theoremstyle{definition}

\renewcommand{\thefootnote}{\fnsymbol{footnote}}

\newcommand{\postscript}[2]
 {\setlength{\epsfxsize}{#2\hsize}
  \centerline{\epsfbox{#1}}}

\renewcommand{\baselinestretch}{1.0}

\setlength{\oddsidemargin}{-0.365in}
\setlength{\evensidemargin}{-0.365in}
\setlength{\topmargin}{-0.3in}
\setlength{\headheight}{0in}
\setlength{\headsep}{0in}
\setlength{\textheight}{10.1in}
\setlength{\textwidth}{7in}
\makeatletter \renewenvironment{proof}[1][Proof] {\par\pushQED{\qed}\normalfont\topsep6\p@\@plus6\p@\relax\trivlist\item[\hskip\labelsep\bfseries#1\@addpunct{.}]\ignorespaces}{\popQED\endtrivlist\@endpefalse} \makeatother
\makeatletter
\renewenvironment{solution}[1][Solution] {\par\pushQED{\qed}\normalfont\topsep6\p@\@plus6\p@\relax\trivlist\item[\hskip\labelsep\bfseries#1\@addpunct{.}]\ignorespaces}{\popQED\endtrivlist\@endpefalse} \makeatother

\begin{document}
\noindent

%========================================================================
\noindent\framebox[\linewidth]{\shortstack[c]{
\Large{\textbf{Lab00-Proof}}\vspace{1mm}\\
CS214-Algorithm and Complexity, Xiaofeng Gao, Spring 2021.}}
\begin{center}
\footnotesize{\color{red}$*$ If there is any problem, please contact TA Haolin Zhou.}

% Please write down your name, student id and email.
\footnotesize{\color{blue}$*$ Name:Renyang Guan  \quad Student ID:519021911058 \quad Email: guanrenyang@sjtu.edu.cn}
\end{center}

\begin{enumerate}
    \item
    Prove that for any integer $n>2$, there is a prime $p$ satisfying $n<p<n!$. {\color{blue}(Hint: consider a prime factor $p$ of $n!-1$ and prove by contradiction)}
    \begin{proof}
    ~\\
    \textit{Case 1:} $p$ is a prime factor of $n!-1$, then $p$ is a prime factor of $n!$. The statement is obviously true.  
    
	\textit{Case 2:} $n!-1$ has no prime factor, then $n!-1$ is a prime number and $n-1 \geq 1$, such that $p=n!-1$ is a prime factor of $n!$.
    \end{proof}

    \item
    Use the minimal counterexample principle to prove that for any integer $n\ge 7$, there exists integers $i_n\ge 0$ and $j_n\ge 0$, such that $n = i_n \times 2 + j_n \times 3$.
    \begin{proof}We proof the statement is true for $n\geq7$ by induction.  
    
    \textbf{Basis step:} 
     ~\\
    $n=7$: $\exists  i_n=2,j_n=1,\quad st. n=i_n\times2 + j_n 	\times 3$ \\
    $n=8$: $\exists  i_n=1,j_n=2,\quad st. n=i_n\times2 + j_n 	\times 3$ \\
    $n=9$: $\exists  i_n=0,j_n=3,\quad st. n=i_n\times2 + j_n 	\times 3$ \\
    $n=10$: $\exists  i_n=2,j_n=2,\quad st. n=i_n\times2 + j_n 	\times 3$ \\
    \textbf{Induction Hypothesis:} Assume that exists $n>10$, there any intergers $i_n\geq 0$ and $j_n\geq 0$, such that $n \neq i_n \times 2+j_n \times 3$  
    
    \textbf{Proof of Induction Step:} Define a new variable $m$,
   	\begin{equation}
	m=\left\{
	\begin{aligned}
	n-2 && i_n\neq 0  \\
	n-3 && i_n = 0 \\
	\end{aligned}
	\right.
	\end{equation}  
	Because $n\geq 10$, therefore $m\geq 7$ and $m<n$ and 
	\begin{equation}
	m\neq \left\{
	\begin{aligned}
	(i_n-1)\times 2+j_n\times 3 && i_n\neq 0 \\
	i_n\times 2+(j_n-1)\times 3 && i_n = 0 \\
	\end{aligned}
	\right.
	\end{equation}	 
	$i_n-1$ and $j_n-1$ can take arbitrary values in $\mathbb{N}$.
    \end{proof}

    \item
    Suppose the function $f$ be defined on the natural numbers recursively as follows: $f(0)=0$, $f(1)=1$, and $f(n)=5f(n-1)-6f(n-2)$, for $n\geq 2$. Use the strong principle of mathematical induction to prove that for all $n\in N$, $f(n)=3^n-2^n$. 
    \begin{proof}We proof $f(n)=3^n-2^n$ is true for $n\geq2$ by induction.  
    
        \textbf{Basis step:} For $n=0,\quad f(n)=3^0-2^0=0$.  
        
        \textbf{Induction Hypothesis:} Let $n>0$. Assume that for any integer $k \in [0,n]$ and $k>0$, such that $f(k)=3^k-2^k$.  
        
        \textbf{Proof of Induction Step:} Now let us prove that $f(n+1)=3^{n+1}-2^{n+1}$ is true.
        \begin{align}
       f(n+1) &=5f(n)-6f(n-1)\\
        & =5(3^{n}-2^{n})-6(3^{n-1}-2^{n-1})\\
        & =9\times 3^{n-1}-4\times 2^{n-1}\\
        & =3^{n+1}-2^{n+1}
        \end{align}
    \end{proof}

    \item
    An $n$-team basketball tournament consists of some set of $n\geq2$ teams. Team $p$ beats team $q$ iff $q$
does not beat $p$, for all teams $p\neq q$. A sequence of distinct teams $p_{1}$, $p_{2}$,..., $p_{k}$, such that team $p_{i}$ beats team $p_{i+1}$ for $1\leq i<k$ is called a ranking of these teams. If also team $p_{k}$ beats team $p_{1}$, the ranking is called a \emph{k-cycle}. 

Prove by mathematical induction that in every tournament, either there is a ``champion" team that beats every other team, or there is a 3-cycle. 
    \begin{proof} We prove the statement is true by induction.
   	~\\
   	\textbf{Basis step:} n=2: Suppose that team team $p$ and team $q$ are in the tournament. Because team $p$ beats team $q$ iff $q$
does not beat $p$, the winner is the champion.  

	\textbf{Induction Hypothesis:} Assume that for any tournament having $n$ teams, either there is a ``champion" team that beats every other team, or there is a 3-cycle.  
	
	\textbf{Proof of Induction Step:} For any tournament having $n+1$ teams denoted as $p_1,p_2\dots , p_{n},p_{n+1}$ there are two cases:
	
	\textit{Case 1:} There is a team denoted as $p_n$ that beats all other $n-1$ teams in $\{p_1,p_2 \dots , p_{n}\}$. If $p_n$ beats $p_{n+1}$, then $p_n$ is the champion. If $p_{n+1}$ beats $p_{n}$, there are two cases:  
	
	\textit{Case 1.1:} $p_{n+1}$ beats all of the teams in $\{p_1,p_2\dots , p_{n-1}\}$, then $p_{n+1}$ is the champion.
	
	\textit{Case 1.2:} Exists $p_i$ in $\{p_1,p_2 \dots , p_{n-1}\}$, such that $p_i$ beats $p_{n+1}$. Then $p_n, p_{n+1},and \quad p_i$ consists of a 3-cycle.
	
	\textit{Case 2:} There is a 3-cycle in $\{p_1,p_2 \dots , p_{n}\}$. Such 3-cycle will be maintained after the participation of team $p_{n+1}$ .
    \end{proof}

\end{enumerate}

\vspace{20pt}

\textbf{Remark:} You need to include your .pdf and .tex files in your uploaded .rar or .zip file.

%========================================================================
\end{document}
