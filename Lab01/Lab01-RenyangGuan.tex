\documentclass[12pt,a4paper]{article}
\usepackage{amsmath,amscd,amsbsy,amssymb,latexsym,url,bm,amsthm}
\usepackage{epsfig,graphicx,subfigure}
\usepackage{enumitem,balance}
\usepackage{wrapfig}
\usepackage{mathrsfs,euscript}
\usepackage[usenames]{xcolor}
\usepackage{hyperref}
\usepackage{booktabs}
\usepackage[vlined,ruled,linesnumbered]{algorithm2e}
\usepackage{threeparttable}
\hypersetup{colorlinks=true,linkcolor=black}

\newtheorem{theorem}{Theorem}
\newtheorem{lemma}[theorem]{Lemma}
\newtheorem{proposition}[theorem]{Proposition}
\newtheorem{corollary}[theorem]{Corollary}
\newtheorem{exercise}{Exercise}
\newtheorem*{solution}{Solution}
\newtheorem{definition}{Definition}
\theoremstyle{definition}

\renewcommand{\thefootnote}{\fnsymbol{footnote}}

\newcommand{\postscript}[2]
 {\setlength{\epsfxsize}{#2\hsize}
  \centerline{\epsfbox{#1}}}

\renewcommand{\baselinestretch}{1.0}

\setlength{\oddsidemargin}{-0.365in}
\setlength{\evensidemargin}{-0.365in}
\setlength{\topmargin}{-0.3in}
\setlength{\headheight}{0in}
\setlength{\headsep}{0in}
\setlength{\textheight}{10.1in}
\setlength{\textwidth}{7in}
\makeatletter \renewenvironment{proof}[1][Proof] {\par\pushQED{\qed}\normalfont\topsep6\p@\@plus6\p@\relax\trivlist\item[\hskip\labelsep\bfseries#1\@addpunct{.}]\ignorespaces}{\popQED\endtrivlist\@endpefalse} \makeatother
\makeatletter
\renewenvironment{solution}[1][Solution] {\par\pushQED{\qed}\normalfont\topsep6\p@\@plus6\p@\relax\trivlist\item[\hskip\labelsep\bfseries#1\@addpunct{.}]\ignorespaces}{\popQED\endtrivlist\@endpefalse} \makeatother

\begin{document}
\noindent

%========================================================================
\noindent\framebox[\linewidth]{\shortstack[c]{
\Large{\textbf{Lab01-Algorithm Analysis}}\vspace{1mm}\\
CS214-Algorithm and Complexity, Xiaofeng Gao, Spring 2021.}}
\begin{center}
\footnotesize{\color{red}$*$ If there is any problem, please contact TA Haolin Zhou. Also please use English in homework.}

% Please write down your name, student id and email.
\footnotesize{\color{blue}$*$ Name:Renyang Guan  \quad Student ID:519021911058 \quad Email: guanrenyang@sjtu.edu.cn}
\end{center}

\begin{enumerate}


\item \textit{Complexity Analysis.} Please analyze the time and space complexity of Alg.~\ref{Alg-quicksort} and Alg.~\ref{Alg-cocktailsort}. \par

\begin{minipage}[t]{0.45\textwidth}
	\begin{algorithm}[H]
		\KwIn{An array $A[1,\cdots,n]$}
		\KwOut{$A[1,\cdots,n]$ sorted nondecreasingly}
		
		\BlankLine
		\caption{QuickSort}\label{Alg-quicksort}
		
		%\If{$n \le 1$}{
		%  \Return\;
		%}
		
		$pivot \leftarrow A[n]$; $i \leftarrow 1$\;
		\For{$j \leftarrow 1$ \KwTo $n-1$}{
			\If{$A[j] < pivot$}{
				swap $A[i]$ and $A[j]$\;
				$i \leftarrow i+1$\;
			}
		}
		
		swap $A[i]$ and $A[n]$\;
		\lIf{$i>1$}{$\operatorname{QuickSort}(A[1,\cdots,i-1])$}
		\lIf{$i<n$}{$\operatorname{QuickSort}(A[i+1,\cdots,n])$}
	\end{algorithm}
\end{minipage}
\hfill
\begin{minipage}[t]{0.45\textwidth}
\begin{algorithm}[H]
\KwIn{An array $A[1,\cdots,n]$}
\KwOut{$A[1,\cdots,n]$ sorted nonincreasingly}
\BlankLine
\caption{CocktailSort}
\label{Alg-cocktailsort}
\BlankLine
	$i\leftarrow 1;$ $j\leftarrow n;$$sorted\leftarrow false$\;
	\While{\textbf{not} sorted}{
		$sorted \leftarrow true$\;
		\For{$k\leftarrow i$ \textbf{to} $j-1$}{
			\If{$A[k] < A[k+1]$}{
				swap $A[k]$ and $A[k+1]$\;
				$sorted\leftarrow false$\;
			}
		}
		$j\leftarrow j-1$\;
		

		\For{$k\leftarrow j$ \textbf{downto} $i+1$}{
			\If{$A[k-1] < A[k]$}{
				swap $A[k-1]$ and $A[k]$\;
				$sorted\leftarrow false$\;
			}
		}
		$i\leftarrow i+1$\;
	}
\end{algorithm}
\end{minipage}

\begin{enumerate}
	 
\item Fill in the blanks and \textbf{explain} your answers. You need to answer when the best case and the worst case happen. 
\begin{table}[!h]

\label{Tab-compare}
	\centering
	\begin{threeparttable}
	\begin{tabular}{c|c| c }
		\toprule[2pt]
		\textbf{Algorithm} & \textbf{Time Complexity}\tnote{1} & \textbf{Space Complexity} \\
		\hline
		\hline
		$QuickSort$ & $\Theta(n\log n)$ \quad $\Theta(n\log n)$ \quad $\Theta(n^2)$ & $O(\log n)$ \\
		
		$CocktailSort$ &$\Theta(n)$\quad \quad\quad $\Theta(n^2)$\quad\quad\quad $\Theta(n^2)$  & $O(1)$  \\
		\bottomrule[2pt]


	\end{tabular}
    \begin{tablenotes}
    	\footnotesize
    	\item[1] The response order can be given in \emph{best}, \emph{average}, and \emph{worst}.
    \end{tablenotes}
	\end{threeparttable}
\end{table}

{\large \textbf{Quick Sort}:}~\\

\textbf{\textit{Best case time complexity:}} 

The best case occurs when every time the partition seperates the sequence with $n$ elements into two subsequences, one of which has $\left \lfloor \frac{n}{2} \right \rfloor$ elements and the other has $\left \lceil \frac{n}{2} \right \rceil -1$ elements.\\
For the sake of convenience,  let $n$ to be an even number. When $n$ is an odd number, the equation is similar, however.

Assume that $T(n)$ is the time complexity of quicksorting a sequence with $n$ elements. It is obviously that the time complexity of the \textbf{\textit{for}} loop is $\Theta(n)$.

Thus, 
$$T(n)=2T(n/2)+\Theta(n)$$ 
According to the master theory, the solution of the equation is $$T(n)=\Theta(n)$$.


\textbf{\textit{Worst case time complexity:}} The worst  case occurs when every time the partition seperates the sequence with $n$ elements into two subsequences, one of which has $0$ elements and the other has $n-1$ elements.

Continue to use the same notation $T(n)$. In this case, 
$$T(n)=T(0)+T(n+1)+\Theta(n)$$
We could get the solution of the equation by substitute method:
$$T(n)=\Theta(n^2)$$

\textbf{\textit{Average case time complexity:}} 
Assume that probabilities of all of the permutations are equal.

Suppose that every partition seperates the array into two arrarys with $i$ and $n-i-1$ elements. The time complexity of partition is $\Theta(n)$, so 
$$T(n)=\frac{1}{n} \sum_{i=0}^{n-1} [T(i)+T(n-i-1)+\Theta(n)]$$.

The solution of the equation is $T(n)=\Theta(n \log n)$.

\begin{proof}~\\
Assume that $\Theta(n)=cn$ when $n>n_0$, where $c$ is a positive number and $n_0$ is a positive integer.
		\begin{align}
       (n+1)T(n+1) &=\sum_{i=0}^{n} [T(i)+T(n-i)+c(n+1)]\\
       & = \sum_{i=0}^{n-1}[T(i)+T(n-i-1)+c(n)]+2T(n)+c(2n+1)\\
       & = (n+2)T(n)+c(2n+1)\\
        \end{align}
       \begin{equation}
       \begin{split}
      & \Rightarrow
       \frac{T(n+1)}{n+2}-\frac{T(n)}{n+1}=\frac{c(2n+1)}{n^2+3n+2}\leq \frac{1}{n+1}
       \\
      & \Rightarrow
       \frac{T(n+1)}{n+2}\leq \sum_{i=1}^{n+1} \frac{1}{i} =\Theta(\log n)
      % \Rightarrow
       T(n+1)=O(n\log n)
	   \end{split}       
       \end{equation}
      Since the average time complexity can not exceed the restriction of best case time complexity which is $\Theta(n)$, thus$$T(n+1)=\Theta(n\log n)$$\end{proof}
\textbf{\textit{Space complexity:}}

The space complexity of every lawyer of the recursion tree is $O(1)$. The progress of recursion will cost stack space which equals to $O(\log n)$. 

Thus, the space complexity of QuickSort is $O(\log n)$.
\\
\\
{\large \textbf{\textbf{Cocktail Sort:}}}~
\\
\\
\textbf{\textit{Best case time complexity:}} The best case occurs when the sequence has been \emph{\textbf{nonincreasingly sorted}}. The \textbf{\textit{while}} loop is only excuted once and the two \textbf{\textit{for}} loop is excuted $2n-3$ ($(n-1)+(n-2)$) times. Thus, the \textbf{best case} time complexity is $\Theta(n)$.
\\
\\
\textbf{\textit{Worst case time conplexity:}} The worst case occurs when the sequence has been \emph{\textbf{nondecreasing sorted}}.  

The worst case time is 
$$\left\lfloor \frac{n}{2} \right\rfloor \times \sum_{i=1}^{\left\lfloor \frac{n}{2} \right\rfloor-1}2\times(n-1-2i)$$
For the sake of convenivence, assume that $n=2k$, where $k$ is any positive integer, such that $\left\lfloor \frac{n}{2} \right\rfloor=k$. Then the worst case time is $$\frac{n^2}{2}-2n+2$$.

Thus, the \textbf{worst case} time complexity is $\Theta(n^2)$.
\\
\\
\textbf{\textit{Average case time complexity:}} Two assumptions:
\begin{enumerate}
\item $A[1,\cdots,n]$contains the numbers 1 through n.
\item All $n!$ permutations are equally likely.
\end{enumerate}

Suppose $A[i]$ should be swapped at position $j(1\leq j \leq n)$.  \\

When $1\leq j\leq i$, we need $i-j$ comparisons to put $A[i]$ at position $j$.  Since and integer in $[1,\cdots,i]$ is equally likely to be taken by j,i.e.,
$$P(j=1)=P(j=2)=\cdots=p(j=i)=\frac{1}{i}$$

When $i< j\leq n$, we need $j-i$ comparisons to put $A[i]$ at position $j$.  Since and integer in $[i+1,\cdots,n]$ is equally likely to be taken by j,i.e.,
$$P(j=n+1)=P(j=n+2)=\cdots=p(j=n)=\frac{1}{n-i}$$

The expectation number of comparisons for put element $A[i]$ in its proper position, is
$$\sum_{j=1}^i \frac{i-j}{i}+\sum_{j=i+1}^n\frac{j-i}{n-i}=\frac{n}{2}$$

The average number of comparisons performed by Alforithm CocktailSort is 
$$\sum_{i=1}^n \frac{n}{2}=\frac{n^2}{2}$$

Thus, the \textbf{average case} complexity is $\Theta(n^2)$

\textbf{\textit{Space complexity:}}

All the operations related to array A are completed in the memory of array A, so no extra space is needed. Thus, the space complexity of CocktailSort is $O(1)$.


\item For Alg.~\ref{Alg-quicksort}, how to modify the algorithm to achieve the same expected performance as the \textbf{average} case when the \textbf{worst} case happens?
\end{enumerate} 
    \begin{solution}
    ~\\
    Considering the best case, an intuitive optimization is to try to seperate the unsorted sequence into to two subsequence nearly equally. The key point is to select an appropriate $pivot$ when the time complexity of partition is $O(nlogn)$

	\textbf{\textit{Optimization method:}}
	\begin{enumerate}
	\item Using the principle of \emph{CountingSort}, store the number of times the integer i appears in the unsorted sequence, where $i$ is any number between $1$ to the maximum number of the unsorted sequence. 
	\item Use a \emph{\textit{for}} loop tranversing the array $B$ which stores the counting result and find the {\large \textbf{\emph{median number}}} to be the $pivot$. 
	\end{enumerate}
	\textbf{\textit{Explanation of the pseudocode:}}~\\
	Line$1-4$ is the progress of finding the maximum of the unsorted array $A$.
	
	Line$5-7$ is to count the total times of integer $i$ appearing in the arry $A$, which is stored in $B[i]$.
	
	Line$8-13$ is to find the $pivot$ which seperate the array $A$ equally.
	
	Line$14-21$ is the common step of \textit{QuickSort}.
	
	\textbf{\textit{Time complexity analysis:}}
	
	The time complexity of the four \emph{\textit{for}} loop is $\Theta(n)$, so we could easily get this equation:
	$$T(n)=2T(n/2)+\Theta(n)$$
	and the solution is
	$$T(n)=\Theta(n\log n)$$
	
	
    \begin{algorithm}[H]
		\KwIn{An array $A[1,\cdots,n]$}
		\KwOut{$A[1,\cdots,n]$ sorted nondecreasingly}
		
		\BlankLine
		\caption{QuickSort-Optimized for the worst case}\label{Alg-quicksort-optimized}
		
		%\If{$n \le 1$}{
		%  \Return\;
		%}
		
		$max\leftarrow 0$\;
		\For{$k \leftarrow 1$ \KwTo $n$}{
			\If{$A[k] > max$}{
				$max\leftarrow A[k]$\;
			}
		}
		\BlankLine
		
		Initialize an array $B[1,\cdots,max]$\;
		\For{$t\leftarrow 1$ \KwTo $n$}{
			$B[A[t]]\leftarrow B[A[t]]+1$\;
		}
		\BlankLine
		$sum\leftarrow 0$\;
		\For{$l \leftarrow 1$ \KwTo $max$}{
			$sum\leftarrow sum +B[l]$\;
			
			\If{$sum> \frac{n}{2}$}{
			$pivot=B[l]$\;
			Break\;
			}		
		}
		\BlankLine
		
		$i \leftarrow 1$\;
		\For{$j \leftarrow 1$ \KwTo $n-1$}{
			\If{$A[j] < pivot$}{
				swap $A[i]$ and $A[j]$\;
				$i \leftarrow i+1$\;
			}
		}
		
		swap $A[i]$ and $A[n]$\;
		\lIf{$i>1$}{$\operatorname{QuickSort}(A[1,\cdots,i-1])$}
		\lIf{$i<n$}{$\operatorname{QuickSort}(A[i+1,\cdots,n])$}
	\end{algorithm}
    \end{solution}

\item \textit{Growth Analysis.} Rank the following functions by order of growth with brief explanations: that is, find an arrangement $g_1, g_2, \ldots , g_{15}$ of the functions $g_1 = \Omega(g_2), g_2 = \Omega(g_3), \ldots, g_{14} = \Omega(g_{15})$.  Partition your list into equivalence classes such that functions $f(n)$ and $g(n)$ are in the same class if and only if $f(n) = \Theta(g(n))$. Use symbols ``$=$'' and ``$\prec$'' to order these functions appropriately. Here $\log n$ stands for $\ln n$.
$$
\begin{array}{ccccc}
	1 \quad & \quad n \quad & \quad \log n \quad & \quad \log (\log n) \quad & \quad n \log n \\
	\log_4 n \quad & \quad 2^n \quad & \quad 4^n \quad & \quad 2^{\log n} \quad & \quad 2^{2^n} \\
	\log (n!) \quad & \quad n! \quad & \quad (2n)! \quad & \quad  n^{1/2} \quad & \quad n^2 \\
\end{array}
$$
	
 	\begin{solution}
	The Ranking of functions is as follows:
	\begin{small}
	$$1 \prec \log (\log n)) \prec \log_4 n = \log n \prec 	\sqrt{n} \prec 2^{\log n} \prec n \prec \log n! = n\log n \prec n^2 	 \prec 2^n \prec 4^n \prec n! \prec 		(2n)! 	\prec 2^{2^n}$$
	\end{small}    
    \textbf{Proof.} I will prove the above relationship in order from left to right \\
    \textit{$1<\log (\log n))$:}
    
    $$\lim_{n\rightarrow\infty} \frac{1}{n}=0$$\\
    \textit{$\log (\log n)) \prec \log_4 n$:}
    $$\lim_{n\rightarrow\infty} \frac{\log (\log n))}{\log_4 n}=\lim_{n\rightarrow\infty} \log 4 \times \frac{\log n}{n}=0 $$
    \textit{$\log_4 n = \log n$:}\\
    $$\lim_{n\rightarrow\infty} \frac{\log_4 n}{\log n}=\frac{1}{\log 4}$$
    \textit{$\log n \prec \sqrt{n}$:}\\
    $$\lim_{n\rightarrow\infty} \frac{\log n}{\sqrt{n}}=\lim_{n\rightarrow\infty} \frac{\frac{1}{n}}{\frac{1}{2\sqrt{n}}}=0$$
    \textit{$\sqrt{n} \prec 2^{\log n}$:} \\
    $$\lim_{n\rightarrow\infty} \frac{\sqrt{n}}{2^{\log n}}=\lim_{n\rightarrow\infty} \frac{n^{\frac{1}{2}}}{n^{\log 2}}=\lim_{n\rightarrow\infty} \frac{1}{n{\log 2-\frac{1}{2}}}=0$$
  	\textit{$2^{\log n}\prec n$:}
  	$$\lim_{n\rightarrow\infty} \frac{2^{\log n}}{n}=\lim_{n\rightarrow\infty}\frac{1}{n^{1-\log 2}}=0$$
   	\textit{$n \prec \log n!$:} Considering $n!=\omega(a^n)$(I will prove it later), where $a$ is a constant integer and $a>1$ \\
   	$$0\leq \lim_{n\rightarrow\infty} \frac{n}{\log n!}\leq \lim_{n\rightarrow\infty} \frac{1}{\log \exp(n)}=\lim_{n\rightarrow\infty} \frac{1}{n}=0 \Rightarrow \lim_{n\rightarrow\infty} \frac{n}{\log n!} =0$$
  	\textit{$\log n! = n\log n$: }
	$$\lim_{n\rightarrow\infty} \frac{\log n!}{n\log n} =1$$
	\textit{$n\log n \prec n^2$:}
	$$\lim_{n\rightarrow\infty} \frac{n\log n}{n^2}=\lim_{n\rightarrow\infty} \frac{1}{n}=0$$
	\textit{$n^2 	\prec 2^{n}$:}
	$$\lim_{n\rightarrow\infty} \frac{n^2}{2^{n}}=0$$
	\textit{$2^n\prec 4^n$:}
	$$\lim_{n\rightarrow\infty} \frac{2^n}{4^n}=\lim_{n\rightarrow\infty} \frac{1}{2^n}=0$$
	\textit{$4^n\prec n!$:} \\In this part, I will prove that for any positive integer $a$, where $a>1$, such that $\lim_{n\rightarrow\infty} \frac{a^n}{n!}=0$ 
	$$0\leq \lim_{n\rightarrow\infty} \frac{a^n}{n!}=\lim_{n\rightarrow\infty} \frac{a^n}{a!\times (a+1)\times \cdots \times n} \leq \lim_{n\rightarrow\infty}\frac{a^2}{a!}\times \frac{a}{n}=0$$
	\textit{$n!\leq (2n)!$:}
	$$\lim_{n\rightarrow\infty} \frac{n!}{(2n)!}=\lim_{n\rightarrow\infty} \frac{1}{(n+1)\times(n+2)\times \cdots	\times 2n}=0$$
	\textit{$(2n)!\leq 2^{2^n}$:}
	
	Let $a_n=\frac{(2n)!}{2^{2^n}}$,
	$$\frac{a_{n+1}}{a_n}=\frac{(2n+1)(2n+2)}{2^{2^{n}}}\leq \frac{1}{2}$$
	Thus, $a_n<(\frac{1}{2})^n$ and 
	$$\lim_{n\rightarrow\infty} \frac{(2n)!}{2^{2^n}}=\lim_{n\rightarrow\infty} a_n=0$$
    \end{solution}


\end{enumerate}

\vspace{20pt}

\textbf{Remark:} You need to include your .pdf and .tex files in your uploaded .rar or .zip file.

%========================================================================
\end{document}
