\documentclass[12pt,a4paper]{article}
\usepackage{amsmath,amscd,amsbsy,amssymb,latexsym,url,bm,amsthm}
\usepackage{epsfig,graphicx,subfigure}
\usepackage{enumitem,balance}
\usepackage{wrapfig}
\usepackage{mathrsfs,euscript}
\usepackage[usenames]{xcolor}
\usepackage{hyperref}
\usepackage[vlined,ruled,linesnumbered]{algorithm2e}
\usepackage{float}
\hypersetup{colorlinks=true,linkcolor=black}

\newtheorem{theorem}{Theorem}
\newtheorem{lemma}[theorem]{Lemma}
\newtheorem{proposition}[theorem]{Proposition}
\newtheorem{corollary}[theorem]{Corollary}
\newtheorem{exercise}{Exercise}
\newtheorem*{solution}{Solution}
\newtheorem{definition}{Definition}
\theoremstyle{definition}

\renewcommand{\thefootnote}{\fnsymbol{footnote}}

\newcommand{\postscript}[2]
 {\setlength{\epsfxsize}{#2\hsize}
  \centerline{\epsfbox{#1}}}

\renewcommand{\baselinestretch}{1.0}

\setlength{\oddsidemargin}{-0.365in}
\setlength{\evensidemargin}{-0.365in}
\setlength{\topmargin}{-0.3in}
\setlength{\headheight}{0in}
\setlength{\headsep}{0in}
\setlength{\textheight}{10.1in}
\setlength{\textwidth}{7in}
\makeatletter \renewenvironment{proof}[1][Proof] {\par\pushQED{\qed}\normalfont\topsep6\p@\@plus6\p@\relax\trivlist\item[\hskip\labelsep\bfseries#1\@addpunct{.}]\ignorespaces}{\popQED\endtrivlist\@endpefalse} \makeatother
\makeatletter
\renewenvironment{solution}[1][Solution] {\par\pushQED{\qed}\normalfont\topsep6\p@\@plus6\p@\relax\trivlist\item[\hskip\labelsep\bfseries#1\@addpunct{.}]\ignorespaces}{\popQED\endtrivlist\@endpefalse} \makeatother

\begin{document}

\noindent

%========================================================================
\noindent\framebox[\linewidth]{\shortstack[c]{
\Large{\textbf{Lab04-Matroid}}\vspace{1mm}\\
CS214-Algorithm and Complexity, Xiaofeng Gao, Spring 2021.}}
\begin{center}
\footnotesize{\color{red}$*$ If there is any problem, please contact TA Haolin Zhou.}

% Please write down your name, student id and email.
\footnotesize{\color{blue}$*$ Name:Renyang Guan  \quad Student ID:519021911058 \quad Email: guanrenyang@sjtu.edu.cn}
\end{center}

\begin{enumerate}
\item \textit{Property of Matroid.} 
\begin{enumerate}
	\item
	Consider an arbitrary undirected graph $ G=(V,E) $. Let us define $ M_{G}=(S,C) $ where $ S=E $ and $ C=\left\{I \subseteq E \mid\left(V, E \backslash I\right) \text { is connected}\right\} $. Prove that $ M_{G} $ is a \textbf{matroid}.\par
	    \begin{proof}
	    ~\\
	    \begin{enumerate}
	    \item
	    \textbf{\textit{Hereditary property:}}
	    
		For $\forall A \subset B$, where $\forall B \in C$, 
		$$B\backslash A\supset \emptyset \Rightarrow B\backslash A=\{e_1,e_2,\cdots,e_k\}(k\geq 1)$$
		Since $(V,E\backslash B)$ is connected, for $\forall D \subseteq B\backslash A$: $(V,(E\backslash B)\cup D)$ is connected.
		
		Considering $(V,(E\backslash B)\cup D)\subset (V,E\backslash A)$, $(V,E\backslash A)$ is connected. 
		
		In other words, $A\in C$.
		
		The property of hereditary has been proved.
	    \\
	    \\
	    \item
	    \textbf{\textit{Exchange property:}}
\\
\\
	    Let us figure out a way to change graph $G_B$ into $G_A=(V,E-A)$.
	    
	    For any edge $e$ in $E$, it could be sorted into one of the categories below:
	    
	    \textbf{Case 1:} $e \notin A\cup B$
	    
	    \textit{No-Changing.}The edge $e$ will remain in $G_A$ because $e\in E-a$.
	    \\
	    \\
	    \textbf{Case 2:} $e \in A\cap B$
	    
	    \textit{No-Changing.}The edge $e$ is excluded from both $G_A$ and $G_B$.
	    \\
	    \\
	    \textbf{Case 3:} $e \in A\backslash B$($e\in B \backslash A$ is discussed in this case in the form of $e'$)
	    
	    \textit{Change.}The edge $e$ will be removed from $G_B$. Such case should be divide into sub-cases depending on whether $e$ is a cut edge.
	    \\
	    \textit{Case 3-1:} $e$ is a cut edge of $G_B$.
	    
	    If $e$ is removed from $G_B$, $G_B$ will be disconnected. There must be one edge $e'\in B\backslash A$ added to the graph, such that $e'$ connects the two connected components which $e$ connected.
	    \\
	    \textit{Case 3-2:} $e$ is not a cut edge of $G_B$.
	    
	 	$e$ ought to be removed from $G_B$ with no need to add a corresponding edge.
	 	\\
	 	\\
	 	\textbf{Let us define such notations:}\\
	    $E_1$ denotes the set of $e$ mentioned in \textit{Case 3-1}.\\
	    $E_2$ denotes the set of $e'$ in corresponding with $e$.
	    \\
	    \\
	    Since $|E_2|=|E_1|\leq |A|<|B|$ and $E_2\subset B\backslash A$, for any edge $x\in B\backslash A-E_2$, $x$ is not a cut edge, which means $E/(A\cup\{x\})$ is connected. In other words,
	    $$A\cup \{x\}\in C$$
	    
	    \end{enumerate}
	    \end{proof}
	\item
	Given a set $A$ containing $n$ real numbers, and you are allowed to choose $k$ numbers from $A$. The bigger the sum of the chosen numbers is, the better. What is your algorithm to choose? Prove its correctness using \textbf{matroid}.\par
	\textbf{Remark:} Denote $\mathbf{C}$ be the collection of all subsets of $A$ that contains no more than $k$ elements. Try to prove $(A,\mathbf{C})$ is a matroid.\par
	    \begin{solution}
	    \textbf{\textit{My algorithm to choose:}}
	    
	\begin{algorithm}[H]
		\KwIn{$A[1,\cdots,n]$}
		\KwOut{$A[1,\cdots,k]$ }
		
		\BlankLine
		\caption{My Algorithm}\label{Alg-my_algorithm}
		
		Sort all elements in $A$ into ordering $a_1\leq a_2\leq \cdots \leq x_n$\;
		$T\leftarrow \emptyset$\;
		\For{$i\leftarrow 1$ \KwTo $n$}{
			\If{$|A\cup \{x_i\}|\leq k$}{
			$A\leftarrow A\cup \{x_i\}$\;			
			}
				
		}
	\end{algorithm}
	    \begin{proof}[Prove that $(A,C)$ is a matroid.]
	    ~\\
	    \begin{enumerate}
	   	\item
	   	\textbf{\textit{Hereditary property:}}\\
	   	$\forall X\subset Y,\quad Y\in C$, since $|Y|\leq k$:
	   	$$|X|\leq k$$
	   	Because $Y\subset A$, such that $$X\subset A$$
	   	Combining the two formulas above, $$X\in C$$.\\
	    The property of hereditary has been proved.
			   	
	   	\item
	   	\textbf{\textit{Exchange property:}}\\
	   	$\forall X,Y\in C$ and $|X|<|Y|$, such that
	   	$$|X|<|Y|\leq k$$
	   	Denote $D$ as
	   	$$D=Y\backslash X=\{d_1,d_2,\cdots,d_k \}(k\leq 1)$$
	   	$\forall d\in D$: $$|X\cup \{d\}|\leq k$$
	   	Because $Y\subseteq A$ and $ \{d\}\subseteq Y$:
	   	$$X\cup \{d\}\subseteq A$$
	   	Combining the formulas above: $$X\cup \{d\}\in C$$
	   	The property of exchange has been proved yet.
	    \end{enumerate}
	    \end{proof}
	    \end{solution}

\end{enumerate}
\item \textit{Unit-time Task Scheduling Problem.} Consider the instance of the \textbf{Unit-time Task Scheduling Problem} given in class. 
    \begin{enumerate}
        \item Each penalty $\omega_{i}$ is replaced by $80-\omega_{i}$. The modified instance is given in Tab.~\ref{tab:1}. Give the final schedule and the optimal penalty of the new instance using Greedy-MAX.
		\begin{table}[H]
			\setlength{\abovecaptionskip}{0.cm}
			\setlength{\belowcaptionskip}{0.5cm}
			\centering
			\caption{Task}
			\label{tab:1}			
			\begin{tabular}{|c|ccccccc|}
				\hline
				$ a_{i} $&1&2&3&4&5&6&7\\
				\hline
				$ d_{i} $&4&2&4&3&1&4&6\\
                \hline
                $ \omega_{i} $&10&20&30&40&50&60&70\\
				\hline
			\end{tabular}
		\end{table}
	        \begin{solution}
	        ~\\
	            The final schedule is $<a_5,a_6,a_4,a_3,a_7,a_1,a_2>$.
	            
	            The optimal penalty is $\omega_1+\omega_2=30$.
	       
	        \end{solution}
        \item Show how to determine in time $O(|A|)$ whether or not a given set $A$ of tasks is independent. (\textbf{Hint}: You can use the lemma of equivalence given in class)
 	        \begin{solution}
 	        ~\\
			\textbf{\textit{Lemma of Equivalence:}} \\
			That the set $A$ is independent is equivalent to that for $t=0,1,2,\cdots,n,N_t(A)\leq t$, where $N_t(A$) denotes the number of tasks in $A$ whose deadline is $t$ or earlier.
			\\
			\\
			\textbf{\textit{The algorithm to determine whether $A$ is dependent in time $O(|A|)$}}\\
			\begin{algorithm}[H]
			\KwIn{A set $A$ of tasks}
			\KwOut{$independent$ or $unindependent$}
			
			\BlankLine
			\caption{Algorithm of determining in time $O(|A|)$}\label{Alg-DetermineIndpendent}
			
			$n\leftarrow |A|$\;
			$N[n+1]\leftarrow=\{0,0,\cdots,0\}$\;
			
			\For{$i\leftarrow 1$ \KwTo $n$}{
				$d_i\leftarrow deadline$ $of$ $task$ $A[i]$\;
				\If{$d_i\leq n$}{
				$N[d_i]\leftarrow N[d_i]+1$\;
				}			
			}
			\For{$t\leftarrow 1$ \KwTo $n$}{
				$N[t]\leftarrow N[t-1]+N[t]$\;
				\If{$N[t]>t$}
				{\Return{$unindependent$}}
			}			
			\Return{$independent$}

			\end{algorithm}
			\textbf{\textit{Description of the algorithm:}}
			\\    	    
			After analysing the property of $N_t(A)$ we could get such recursive formula easily:
			$$			
			N_t(A)=\left\{			
			\begin{aligned}
			&0 &t=0 \\
			&N_{t-1}(A)+number\ of\ tasks\ whose\ dealine\ is\ t &t\geq 1
			\end{aligned}
			\right.
			$$
			
			The \textbf{for} loop starting in line 3, which loops for $|A|$ times, traverses the set $A$. It set the number of tasks whose deadline is $t$ in $N[t]$.
			
			The \textbf{for} loop starting in line 7, which loops for $|A|$ times, calculate $N_t(A)$ according to the recursive formula above and set $N_t(A)$ to $N[t]$. At the same time, it checks every $N_t(A)$ whether or not it is greater than $t$, where $t$ is an integer ranges from $1$ to $n$.
			
			The total time complexity of the algorithm is $O(|A|)$.
			

            \end{solution}
    \end{enumerate}

\item \textit{MAX-3DM.} Let $X$, $Y$, $Z$ be three sets. We say two triples $\left(x_{1}, y_{1}, z_{1}\right)$ and $\left(x_{2}, y_{2}, z_{2}\right)$ in $X \times Y \times Z$ are \textit{disjoint} if $x_{1} \neq x_{2}$, $y_{1} \neq y_{2},$ and $z_{1} \neq z_{2}$. Consider the following problem:
    
    \begin{definition}[MAX-3DM] 
        Given three disjoint sets $X$, $Y$, $Z$ and a non-negative weight function $c(\cdot)$ on all triples in $X \times Y \times Z$, \textbf{Maximum 3-Dimensional Matching} (MAX-3DM) is to find a collection $\mathcal{F}$ of disjoint triples with maximum total weight.
    \end{definition}

    \begin{enumerate}
    	\item Let $D = X \times Y \times Z$. Define independent sets for MAX-3DM.
    	\item Write a greedy algorithm based on Greedy-MAX in the form of \emph{pseudo code}. \label{Item-Greedy}
    	\item Give a counter-example to show that your Greedy-MAX algorithm in Q.~\ref{Item-Greedy} is not optimal.
    	\item Show that: $\max\limits_{F \subseteq D} \frac{v(F)}{u(F)} \leq 3$. {\color{blue}(Hint: you may need Theorem~\ref{Thm-Intersect} for this subquestion.)} 
    	    \begin{solution}
    	    ~\\
    	    \begin{enumerate}
    	    \item[(a)]
    	    \begin{definition}[$(D,C)$]
    	    Let $A$ denote any subset of $S$, in which any two triples $x,y\in A$, $x$ and $y$ are disjoint. $C$ is the collection of all possible subset $A$s. 
    	    \end{definition}
    	    
    	   	\begin{proof}[Proof of independent property of $(D,C)$]
			~\\    	   	
    	   	$\forall A\subset B$, $\forall B\in C$, $\forall x,y \in A$
    	   	$$x,y \in B\Rightarrow x,y\ are \	disjoint\Rightarrow A\in C$$
    	   	
    	   	\end{proof}
    	    \item[(b)]
    	    ~\\
    	    \begin{algorithm}[H]
    	    \KwIn{A set of triples $D$}
    	    \KwOut{A collection $A$ of disjoint triples}
    	    
    	    \BlankLine
    	    
    	    \caption{Greedy-Max For MAX-3DM}\label{Alg-GreedyMax}
    	    
    	    Sort all elements in $D$ into ordering $c(x_1)\geq c(x_2)\geq \cdots \geq c(x_n)$\;
    	    $A\leftarrow \emptyset$\;
    	    $n\leftarrow |D|$\;
    	    \For{$i\leftarrow 1$ \KwTo $n$}{\
    	    	\If{$A\cup {x_i}\in C$}{
    	    		$A\leftarrow A\cup \{x_i\}$\;
    	    	}    
    	    }
    	    \Return{$A$}
    	    \end{algorithm}
    	 
    	    \item[(c)]
    	    \textbf{\textit{A Counter Example:}}
			\\    	    
    	    $$X=\{1,2\}\quad Y=\{3,4\} \quad Z=\{5,6\}$$
    	    In such case:
    	    $$D=\{(1,3,5),(1,3,6),(1,4,5),(1,4,6),(2,3,5),(2,3,6),(2,4,5),(2,4,6)\}$$
    	    The function $C(*)$ is shown below:
    	    \begin{equation}
    	    \begin{split}
    	    & (1,3,5)\rightarrow 9\quad    	     
    	     (2,4,6)\rightarrow 1\\ 
			& (2,4,5)\rightarrow 8\quad
			 (1,3,6)\rightarrow 8\\   	    
    	    & (1,4,5)\rightarrow 1\quad
    	     (1,4,6)\rightarrow 1\\
    	    & (2,3,5)\rightarrow 1\quad
    	     (2,3,6)\rightarrow 1
    	    \end{split}
    	    \end{equation}
    	    It is obvious that the optimal solution is $\mathcal{F}=\{(2,4,5),(1,3,6)\}$ where the maximum is $18$ while the solution of   Greedy-Max algorithm is $\mathcal{F}=\{(1,3,5),(2,4,6)\}$ where the sum is $10$.
    	    
    	    \item[(d)]
			According to the Theorem.~\ref{Thm-Intersect}, to prove the origin proposition we just need to prove the Lemma.~\ref{Lem-FxIsMatroid} and Lemma.~\ref{Lem-CupOfFxFyFzIsF} below.
			\begin{lemma}\label{Lem-FxIsMatroid}
			~\\
			Let the set $A$ satisfy:
			$$A=\{(x,y,z)|x\in X,y\in Y,z\in Z,\ and \ \forall (x_1,y_1,z_1),(x_2,y_2,z_2)\in A:x_1\neq x_2 \}$$
			and $\mathcal{F}_x$ is the collection of all the subsets of A.
			
			The system $(D,\mathcal{F}_x)$ is a matroid.
			\end{lemma}
			\begin{lemma}\label{Lem-CupOfFxFyFzIsF}
			We define matroid $(D,\mathcal{F}_y)$ and $(D,\mathcal{F}_z)$ similar to that in Lemma.~\ref{Lem-FxIsMatroid}.
			$\mathcal{F}_x$, $\mathcal{F}_y$, and $\mathcal{F}_z$ satisfy:
			$$\mathcal{F}_x\cap \mathcal{F}_y \cap \mathcal{F}_z = \mathcal{F}$$
			\end{lemma}
			\begin{proof}[Proof of Lemma.~\ref{Lem-FxIsMatroid}]
			~\\
			\textbf{\textit{Hereditary property:}}\\
			$\forall B \in \mathcal{F}_x, \forall A\subset B$:
			\begin{equation}\nonumber
			\begin{split}
			& \forall (x_1,y_1,z_1),(x_2,y_2,z_2)\in A\\
			\Rightarrow\ & (x_1,y_1,z_1),(x_2,y_2,z_2)\in B\\
			\Rightarrow\ & x_1\neq x_2\\
			\Rightarrow\ & A\in \mathcal{F}_x
			\end{split}
			\end{equation}
			
			
			\textbf{\textit{Exchange property:}}\\
			$\forall A,B\in \mathcal{F}_x$ and $|A|<|B|$, according to the definition of $\mathcal{F}_x$, we could easily get:
			$$\forall (x_1,y_1,z_1),(x_2,y_2,z_2)\in A:x_1\neq x_2$$
			$$\forall (x_1,y_1,z_1),(x_2,y_2,z_2)\in B:x_1\neq x_2$$
			Let set $A_x=\{x|\ \forall(x,y,z)\in A\}$ and set $B_x=\{x|\ \forall(x,y,z)\in B\}$
			
			Because $|A|<|B|$ and that the $x$ value of every two different triples in $A$ are different,which is similar to $B$, we have: $$|A_x|<|B_x|$$
			As a result of this, $$\exists x \in B_x,\ \forall x' \in A_x:\ x\neq x'$$
			(Otherwise $B_x$ is a subset of $A_x$, which conflicts with $|A|<|B|$.)\\
			and
			$$\forall y\in Y,\ \forall z\in Z, (x,y,z)\notin A$$
			According to the definition of $B_x$:
			$$ \exists (x,y,z)\in B\backslash A,\ \forall (x',y',z')\in A:\ x\neq x' 
			\Rightarrow  A\cup \{(x,y,z)\}\in \mathcal{F}_x $$
			
			\end{proof}
			\begin{proof}[Proof of Lemma.~\ref{Lem-CupOfFxFyFzIsF}]
			~\\
			\textbf{\textit{Every element of $\mathcal{F}$ is in $\mathcal{F}_x$, $\mathcal{F}_y$, and $\mathcal{F}_z$}}
			\\
			\\
			$\forall A \in \mathcal{F},\ \forall (x_1,y_1,z_1),(x_2,y_2,z_2)\in A$: $x_1\neq x_2,y_1\neq y_2,z_1\neq z_2$
			\begin{equation}
			\begin{split}
			& x_1\neq x_2 \Rightarrow A\in \mathcal{F}_x\\
			& y_1\neq y_2 \Rightarrow A\in \mathcal{F}_y\\
			& z_1\neq z_2 \Rightarrow A\in \mathcal{F}_z\\			
			\end{split}
			\Rightarrow A\in \mathcal{F}_x\cap \mathcal{F}_y \cap \mathcal{F}_z
			\end{equation}
			\\
			\textbf{\textit{Any element not in $\mathcal{F}$ is not in $\mathcal{F}_x$(or $\mathcal{F}_y$ or $\mathcal{F}_z$):}}\
			\\
			\\
			$\forall A\notin \mathcal{F},\ \forall (x_1,y_1,z_1),(x_2,y_2,z_2)\in A:$
			$$x_1\neq x_2 \ or\ y_1\neq y_2\ or \ z_1\neq z_2\Rightarrow (A\notin \mathcal{F}_x)\vee (A\notin \mathcal{F}_y )\vee( A\notin \mathcal{F}_z)$$ 
			\end{proof}
    	    \end{enumerate}
    	    \end{solution}
    \end{enumerate}
    \begin{theorem} \label{Thm-Intersect}
        Suppose an independent system $(E, \mathcal{I})$ is the intersection of $k$ matroids $\left(E, \mathcal{I}_{i}\right)$, $1 \leq i \leq k$; that is, $\mathcal{I}=\bigcap_{i=1}^{k} \mathcal{I}_{i}$. Then $\max\limits_{F \subseteq E} \frac{v(F)}{u(F)} \leq k$, where $v(F)$ is the maximum size of independent subset in $F$ and $u(F)$ is the minimum size of maximal independent subset in $F$.
    \end{theorem}    
\end{enumerate}

\vspace{20pt}

\textbf{Remark:} You need to include your .pdf and .tex files in your uploaded .rar or .zip file.

%========================================================================
\end{document}
